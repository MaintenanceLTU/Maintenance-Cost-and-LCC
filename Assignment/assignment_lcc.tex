\documentclass[a4paper,12pt]{exam}
\usepackage{graphicx}
\usepackage{amsmath}
\usepackage{booktabs}
\usepackage{hyperref} % Enables hyperlinks
\usepackage{parskip} % Removes indentation and adds space between paragraphs


\title{LCC Analysis of a Crushing Plant}
\author{Johan Odelius \\
        Division of Operation and Maintenance \\
        Lule\aa{} University of Technology}
\date{2025-02-28}

\begin{document}

\maketitle

\section*{Background}
In this assignment, you are required to perform a Life Cycle Cost (LCC) analysis of a crushing plant. The plant consists of a crusher unit, a conveyor, and a screen adapted for the desired grain size. You are responsible for this production facility and should investigate how various factors affect your LCC.


Tutorials for MATLAB and Python are available, visit:  
\href{https://github.com/MaintenanceLTU/D0002B/tree/main/Maintenance%20Cost%20and%20LCC}{MaintenanceLTU GitHub Repository}. Include your code ass appendix to your report. 
\section{Availability and LCC calculations}
\subsection{Availability Calculations}
Calculate the maintenance time and availability using the following data:
\begin{itemize}
    \item \textbf{Discount rate:} 10\%.
    \item \textbf{Calculation period:} 5 years.
    \item \textbf{Operating time:} 8,424 hours per year (24 hours/day, excluding a two-week annual maintenance stop in August).
    \item \textbf{Corrective maintenance:}
    \begin{itemize}
        \item Number of failures per year: 28.
        \item Mean Time to Repair (MTTR): 0.8 days.
        \item Mean Waiting Time for Corrective Maintenance: 0.2 days.
    \end{itemize}
\end{itemize}
Present the following variables in a table: 
\begin{itemize}
    \item Preventive maintenance time
    \item Repair time
    \item Corrective maintenance time
    \item Downtime
    \item Available time
    \item Operational availability ($A_o$)
\end{itemize}
\subsection{Life Cycle Cost Calculations}
Calculate the life cycle costs of the asset.
\begin{itemize}
    \item Yearly maintenance cost.
    \begin{itemize}
    \item The mean hourly cost for corrective maintenance is \textbf{7,000 SEK}. 
    \item The cost of the two-week preventive maintenance cost is \textbf{1,400,000 tons/year}. 
    \end{itemize}    
    \item Yearly income. 
    \begin{itemize}
    \item The capacity of the crusher plant is \textbf{265,000 tons/year}.
    \item The price of macadam is \textbf{ 100 SEK / ton}.
    \item Include parameters for: \begin{itemize}
        \item Performance (initially set to 1, full capacity).
        \item Quality (initially set to 1, 100\% quality yield).
    \end{itemize}
    \end{itemize}
    \item Yearly operating costs:
    \begin{itemize}
        \item Energy cost: 600 SEK/hour (scalable with performance).
        \item Operator cost: 500 SEK/hour.
    \end{itemize}
    \item Investment cost: 
    \begin{itemize}
    \item The basic price of the asset is \textbf{ 14,000,000 SEK}.
    \end{itemize}
    \item Residual value: 
    \begin{itemize}
    \item Investment reduces by \textbf{50\% per year}.
    \end{itemize}
\end{itemize}

\subsection{Net Present Value (NPV) Calculation}
Calculate present values of annual costs (operation, maintenance, and income) and single amounts (residual value). Then compute NPV and report using the table format in Table~\ref{NPVtable}.

\begin{table}[h]
    \centering
    \begin{tabular}{lccc}
        \toprule
        \textbf{Item} & \textbf{Present Value} & \textbf{Annual Amounts} & \textbf{Amount in Year X} \\
        \midrule
        Investment & -XXX &  &  \\
        Maintenance & -XXX & -XXX & -XXX \\
        Operating Cost & -XXX & -XXX & -XXX \\
        Revenue & XXX & XXX &  \\
        Residual Value & XXX & XXX &  \\
        \textbf{NPV} & \textbf{XXX} &  &  \\
        \bottomrule
    \end{tabular}
    \caption{NPV Calculation Summary}
    \label{NPVtable}
\end{table}

\section{Improving availability}
\subsection{Increasing MTBF}
Perform the same calculations as in Task 1, assuming an additional MTBF. Each \textbf{1\% extra MTBF} increases the investment cost by \textbf{125,000 SEK}. Compute the \textbf{NPV for 10\% extra MTBF} using the same table format (Table~\ref{NPVtable}).

\textbf{Tip:} A 10\% increase in MTBF is modeled as a \textbf{$1/(1.10)$ reduction} in the failure rate.

\subsection{Preventive Maintenance Stops}
Perform the same calculations but include \textbf{preventive maintenance (PM) stops}:
\begin{itemize}
    \item Each PM stop lasts \textbf{24 hours}, costs \textbf{140,000 SEK}, and reduces failures by \textbf{5\%}.
    \item Update failure count, PM time, and maintenance costs.
\end{itemize}
Compute NPV for\textbf{6 additional PM stops} and report using the same table format.

\section{Maintenance Optimization}
\subsection{Finding Optimal MTBF Increase}
Plot \textbf{NPV vs. extra \% MTBF} for the range \textbf{0–100\%}, in \textbf{1\% steps}, using \textbf{6 extra PM stops}.

\subsection{Finding Optimal PM Stops}
Plot \textbf{NPV vs. number of PM stops} in the range \textbf{0–24 stops}, using \textbf{10\% extra MTBF}.

\textbf{Report:} Present the optimal solution with graphs.

\section{Sensitivity Analysis}
Perform sensitivity analysis using \textbf{10\% extra MTBF and 6 extra PM stops}.

\subsection{Minimum Quality Yield}
Set plant efficiency to 95\%. What is the minimum quality yield needed for a positive NPV? 

\subsection{Break-even Price}
Set plant efficiency to 95\% and quality to 95\%. At what crushed stone price does NPV become negative?

\subsection{Interest Rate Sensitivity}
For a \textbf{stone price of 85 SEK}, what discount rate causes NPV $<0$?

\textbf{Report:} Present results and graphs.

\end{document}
